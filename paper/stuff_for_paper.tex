\documentclass[10pt]{report} %article
%\usepackage{spconf}
%\usepackage{cite}
%\usepackage{psfrag}
\usepackage[dvips]{graphicx}
%\usepackage[caption=false, font=footnotesize]{subfig}
\usepackage{url}
\usepackage[T1]{fontenc}
\usepackage{epsfig}
\usepackage[cmex10]{amsmath}
\usepackage{amssymb} % Necessary for AMS symbols like \square
\usepackage{times}

%%%%%%%%%%%%%%%%%%%%%%%%%%%%%%%%%%%%%%%%%%%%%%%%%%%%%%%
%
% COMMAND FOR COMPILING IN LINUX/LATEX to EMBED ALL FONTS
%
% ps2pdf -dCompatabilityLevel = 1.4 -dPDFSETTINGS=/prepress -dEmbedAllFonts=true
%
%%%%%%%%%%%%%%%%%%%%%%%%%%%%%%%%%%%%%%%%%%%%%%%%%%%%%%%

%\renewcommand{\baselinestretch}{2}

\title{Notes}

%\name{N.J. Stevenson}


\begin{document}

\maketitle

\section{Multicomponent AM-FM signal model}

A monocomponent estimate of the IF of a signal makes little sense applied to multicomponent signals \cite{}. Unless the components are well separated in time \cite{}. A monocomponent estimate of the IF will have a modulation around the mean IF. This modulation is what is seen in 'nonlinear signals' as processed by the EMD. The modulation of the IF is generated by the harmonics of the nonlinear signal.

This following analysis is based on the definition of the IF of a signal as,
\begin{equation}
f_i(t) = \phi'(t)
\end{equation}

An example of this modulation for a complex two tone signal \cite{}. Given a signal,
\begin{equation}
z(t) = A(t)e^{j\phi t} = A_1e^{j\omega_1 t} + A_2e^{j\omega_2 t}
\end{equation}
with a magnitude squared,
\begin{equation}
|z(t)|^2 = A^2(t) = A_1^2 + A^2_2 + 2A_1A_2 \cos(\omega_1 t - \omega_2 t)
\end{equation}
the IF is defined as, 
\begin{equation}
\phi'(t) =\frac{1}{2}(\omega_1 t + \omega_2 t) + \frac{1}{2}(\omega_2 t - \omega_1 t) \frac{A_2^2 - A_1^2}{A^2(t)}
\end{equation}
so when $|A_1| = |A_2|$ the oscillation in the IF disappears, but this is rarely the case with harmonic signals or most real signals.

WORKING
Magnitude of $z(t) \rightarrow Ae^{\omega t} = A\left\( \cos(\omega t) + j \sin(\omega t)\right\)$
\begin{displaymath}
z(t) = = A_1\left\( \cos(\omega_1 t) + j \sin(\omega_1 t)\right\) + A_2\left\( \cos(\omega_2 t) + j \sin(\omega_2 t)\right\)
\end{displaymath}
Magnitude of $z(t) = a+jb = \sqrt{a^2+b^2}, a = A_1 \cos(\omega_1 t)+ A_2 \cos(\omega_2 t), b = A_1 \sin(\omega_1 t)+ A_2 \sin(\omega_2 t)$.
\begin{eqnarray*}
|z(t)|^2 &=& \left\( A_1 \cos(\omega_1 t)+ A_2 \cos(\omega_2 t) \right)^2 + \left\( A_1 \sin(\omega_1 t)+ A_2 \sin(\omega_2 t) \right\)^2 \\
 &=& A_1^2 \cos^2(\omega_1 t)+ A_2^2 \cos^2(\omega_2 t) + 2A_1A_2\cos(\omega_1 t)\cos(\omega_2 t) + A_1^2 \sin^2(\omega_1 t)+ A_2^2 \sin^2(\omega_2 t) + 2A_1A_2\sin(\omega_1 t)\sin(\omega_2 t) \\
   &=& A_1^2 \left\( \cos^2(\omega_1 t)+\sin^2(\omega_1 t) \right\) + A_2^2 \left\( \cos^2(\omega_1 t)+\sin^2(\omega_1 t) \right\) + 2A_1A_2 \left\( \cos(\omega_1 t)\cos(\omega_2 t) + \sin(\omega_1 t)\sin(\omega_2 t) \right\)
\end{eqnarray*}
Noting  $\cos^2(x)+\sin^2(y) = 1$, $2\cos(x)\cos(y) = \cos(x-y)+\cos(x+y)$ and $2\sin(x)\sin(y) = \cos(x-y)-\cos(x+y)$ results in 
\begin{eqnarray*}
   &=& A_1^2 + A_2^2 + A_1A_2 \left\( \cos(\omega_1 t - \omega_2 t) + \cos(\omega_1 t + \omega_2 t) + \cos(\omega_1 t - \omega_2 t) -\cos(\omega_1 t + \omega_2 t)) \right\) \\
   &=& A_1^2 + A_2^2 + A_1A_2 \left\( \cos(\omega_2 t - \omega_1 t) + \cos(\omega_2 t + \omega_1 t) + \cos(\omega_2 t - \omega_1 t) -\cos(\omega_2 t + \omega_1 t)) \right\) \\
   &=& A_1^2 + A_2^2 + 2A_1A_2 \cos(\omega_2 t - \omega_1 t) 
\end{eqnarray*}
Phase of $z(t) = a+jb = \arctan (b/a)$. There are regions of validity for this function, which result in the use of atan2 when programming the arctan function, but I will stick with the original here. The derivative of the phase, which defines the IF, is
\begin{displaymath}
\frac{d \arctan x}{dx} = \frac{1}{1+x^2}
\begin{displaymath}
Solving using the chain rule results in,
\begin{eqnarray*}
\frac{d \phi(t)}{dt} = \frac{d \phi}{dx}\frac{dx}{dt} &=& \frac{1}{1+\left\( frac{b}{a} \right\) ^2} \frac{(b'a-ba')}{a^2} \\
&=& \frac{(b'a-ba')}{a^2+b^2}
\begin{eqnarray*}
where
\begin{eqnarray*}
a' &=& -1 (A_1\omega_1 \sin(\omega_1 t)+ A_2 \omega_2 \sin(\omega_2 t)) 
b' &=& b = A_1\omega_1 \cos(\omega_1 t)+ A_2 \omega_2 \cos(\omega_2 t).
\begin{eqnarray*}
and $a^2+b^2 = A^2(t)$. This results in 
\begin{displaymath}
\frac{d \phi(t)}{dt} = \frac{A_1^2\omega_1 + A_2^2 \omega_2+A_1A_2(\omega_1+\omega_2)\cos(\omega_2t-\omega_1t)}{A^2(t)}
\begin{displaymath}
which also equals,
\begin{displaymath}
\phi'(t) =\frac{1}{2}(\omega_1 t + \omega_2 t) + \frac{1}{2}(\omega_2 t - \omega_1 t) \frac{A_2^2 - A_1^2}{A^2(t)}
\end{displaymath}

Another way to estimate the instantaneous frequency is to use the analytic associate of a signal directly.
\begin{equation}
f_i(t) = \frac{|z'(t)|}{|z(t)|}
\end{equation}

\subsection{Time-varying filtering}

The time-varying FIR bandpass filter is limited. Trade off - high order filters will have a smaller bandwidth but longer burn-in periods, while low order filters will have larger bandwidths. This is shown in Fig. \
 

Knowledge of the filter bandwidth is important when segmenting the time-frequency domain as overlapping segmentations will result in the same TF energy being represented twice in the decomposition; this will increase the reconstruction error. These limits also influences the smoothing window applied to the WVD.



\begin{equation}
h^i(n) = \left\{ \begin{array}{ll}
\frac{1}{\pi(n-\alpha)}(sin(2 \pi f_{c2}^i(n-\alpha))-sin(2\pi f_{c1}^i(n-\alpha))) & \quad  \mathrm{for} \; n \neq \alpha \\
2*(fc2(ii)-fc1(ii)) & \quad \mathrm{for} \; n = \alpha \\
\end{array} \right.
\end{equation}
when M -s odd and 
\begin{equation}
h^i() = 1./(pi.*(n-alpha)).*(sin(2*pi*fc2(ii)*(n-alpha))-sin(2*pi*fc1(ii)*(n-alpha)));  % n~=alpha M = odd
\end{equation}
when $M$ is even, $n = 0,..., M-1$ and $i$ is $0,...,N-1$.





\begin{thebibliography}{1}

{\footnotesize

\bibitem{loug-tace-97} P.J. Loughlin and B. Tacer, Comments on the interpretation of instantaneous frequency. IEEE Signal Processing Letters. 4(5):123-125, May 1997.

\bibitem{wei-bovi-98} D .Wei and A.C. Bovik, On the instantaneous frequencies of multicomponent AM-FM signals. IEEE Signal Processing Letters. 5(4):84-86, April 1998.

}

\end{thebibliography}


\end{document}
